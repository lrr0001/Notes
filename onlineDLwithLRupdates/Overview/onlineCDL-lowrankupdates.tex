\documentclass{article}
\usepackage{amsmath}
\usepackage{bm}
\usepackage{bbm}

% fields
\newcommand{\R}{\mathbb{R}}
\newcommand{\C}{\mathbb{C}}
\newcommand{\Z}{\mathbb{Z}}
\newcommand{\N}{\mathbb{N}}

% complex numbers
\renewcommand{\Re}[1]{\operatorname{Re}\left\{#1\right\}}
\renewcommand{\Im}[1]{\operatorname{Im}\left\{#1\right\}}
\newcommand{\conj}[1]{\mkern 1.5mu\overline{\mkern-1.5mu#1\mkern-1.5mu}\mkern 1.5mu}

% probability and stat
\renewcommand{\P}[1]{\operatorname{P}\left(#1\right)}
\newcommand{\E}{\operatorname{E}}
\newcommand{\var}{\operatorname{var}}

% calculus 
\renewcommand{\d}[1]{d#1}

% constants (written in roman, if wanted)
\newcommand{\e}{e}
\renewcommand{\j}{j}

% linear algebra
% 	vector notation
%\newcommand{\vct}[1]{\boldsymbol{#1}}
\newcommand{\vct}[1]{\bm{#1}}
%   matrices
%\newcommand{\mtx}[1]{\boldsymbol{#1}}
\newcommand{\mtx}[1]{\bm{#1}}
% diagonal
\newcommand{\diag}{\operatorname{diag}}
%   block vector
\newcommand{\bvct}[1]{\mathbf{#1}}
%   block matrix
\newcommand{\bmtx}[1]{\mathbf{#1}}
%	inner products
\newcommand{\<}{\langle}
\renewcommand{\>}{\rangle}
% 	transpose, Hermitian, pseudo-inverse
\renewcommand{\H}{\mathrm{H}}
\newcommand{\T}{\mathrm{T}}
\newcommand{\pinv}{\dagger}
%	fundamental subspaces
\newcommand{\Null}{\operatorname{Null}}
\newcommand{\Range}{\operatorname{Range}}
\newcommand{\Span}{\operatorname{Span}}
%	operators
\newcommand{\trace}{\operatorname{trace}}
\newcommand{\rank}{\operatorname{rank}}
\newcommand{\F}{\mathcal{F}}

% sets and topology
\newcommand{\set}[1]{\mathcal{#1}}
\newcommand{\closure}{\operatorname{cl}}	% closure
\newcommand{\interior}{\operatorname{int}}
\newcommand{\boundary}{\operatorname{bd}}
\newcommand{\diameter}{\operatorname{diam}}

% functional analysis
\newcommand{\domain}{\operatorname{dom}}
\newcommand{\epigraph}{\operatorname{epi}}
\newcommand{\hypograph}{\operatorname{hypo}}
\newcommand{\linop}[1]{\mathscr{#1}}	% general linear operator

% optimization
\renewcommand{\L}{\mathcal{L}}
\DeclareMathOperator*{\minimize}{\text{minimize}}
\DeclareMathOperator*{\maximize}{\text{maximize}}
\newcommand{\indicator}{\mathbbm{1}}
\newcommand{\prox}{\operatorname{prox}}

% Complexity
\newcommand{\bigO}{\mathcal{O}}

% Neural Networks
\newcommand{\pool}{\mtx{P}}


%--------------------------------------------------------------------------


\newcommand{\va}{\vct{a}}
\newcommand{\vb}{\vct{b}}
\newcommand{\vc}{\vct{c}}
\newcommand{\vd}{\vct{d}}
\newcommand{\ve}{\vct{e}}
\newcommand{\vf}{\vct{f}}
\newcommand{\vg}{\vct{g}}
\newcommand{\vh}{\vct{h}}
\newcommand{\vi}{\vct{i}}
\newcommand{\vj}{\vct{j}}
\newcommand{\vk}{\vct{k}}
\newcommand{\vl}{\vct{l}}
\newcommand{\vm}{\vct{m}}
\newcommand{\vn}{\vct{n}}
\newcommand{\vo}{\vct{o}}
\newcommand{\vp}{\vct{p}}
\newcommand{\vq}{\vct{q}}
\newcommand{\vr}{\vct{r}}
\newcommand{\vs}{\vct{s}}
\newcommand{\vt}{\vct{t}}
\newcommand{\vu}{\vct{u}}
\newcommand{\vv}{\vct{v}}
\newcommand{\bvv}{\bvct{v}}
\newcommand{\vw}{\vct{w}}
\newcommand{\vx}{\vct{x}}
\newcommand{\vy}{\vct{y}}
\newcommand{\vz}{\vct{z}}
\newcommand{\bvz}{\bvct{z}}
%
\newcommand{\valpha}{\vct{\alpha}}
\newcommand{\bvalpha}{\bvct{\alpha}}
\newcommand{\vbeta}{\vct{\beta}}
\newcommand{\vepsilon}{\vct{\epsilon}}
\newcommand{\vgamma}{\vct{\gamma}}
\newcommand{\vlambda}{\vct{\lambda}}
\newcommand{\vnu}{\vct{\nu}}
\newcommand{\vmu}{\vct{\mu}}
\newcommand{\bvmu}{\bvct{\mu}}
\newcommand{\vphi}{\vct{\phi}}
\newcommand{\vpsi}{\vct{\psi}}
\newcommand{\vtheta}{\vct{\theta}}
\newcommand{\veta}{\vct{\eta}}
%
\newcommand{\vzero}{\vct{0}}
\newcommand{\vone}{\vct{1}}

%------------------------------------------------

\newcommand{\mA}{\mtx{A}}
\newcommand{\mB}{\mtx{B}}
\newcommand{\mC}{\mtx{C}}
\newcommand{\mD}{\mtx{D}}
\newcommand{\bmD}{\bmtx{D}}
\newcommand{\mE}{\mtx{E}}
\newcommand{\mF}{\mtx{F}}
\newcommand{\mG}{\mtx{G}}
\newcommand{\mH}{\mtx{H}}
\newcommand{\mJ}{\mtx{J}}
\newcommand{\mK}{\mtx{K}}
\newcommand{\mL}{\mtx{L}}
\newcommand{\mM}{\mtx{M}}
\newcommand{\mN}{\mtx{N}}
\newcommand{\mO}{\mtx{O}}
\newcommand{\mP}{\mtx{P}}
\newcommand{\mQ}{\mtx{Q}}
\newcommand{\mR}{\mtx{R}}
\newcommand{\mS}{\mtx{S}}
\newcommand{\mT}{\mtx{T}}
\newcommand{\mU}{\mtx{U}}
\newcommand{\mV}{\mtx{V}}
\newcommand{\mW}{\mtx{W}}
\newcommand{\mX}{\mtx{X}}
\newcommand{\mY}{\mtx{Y}}
\newcommand{\mZ}{\mtx{Z}}
%
\newcommand{\mDelta}{\mtx{\Delta}}
\newcommand{\mLambda}{\mtx{\Lambda}}
\newcommand{\mPhi}{\mtx{\Phi}}
\newcommand{\mPsi}{\mtx{\Psi}}
\newcommand{\mSigma}{\mtx{\Sigma}}
\newcommand{\mUpsilon}{\mtx{\Upsilon}}
\newcommand{\mGamma}{\mtx{\Gamma}}
%
\newcommand{\mId}{{\bf I}}
\newcommand{\mEx}{{\bf J}}
\newcommand{\mzero}{{\bf 0}}
\newcommand{\mone}{{\bf 1}}

\newcommand{\mAbar}{\underline{\mtx{A}}}
\newcommand{\mRbar}{\underline{\mtx{R}}}
\newcommand{\vebar}{\underline{\vct{e}}}
\newcommand{\vxbar}{\underline{\vct{x}}}
\newcommand{\vybar}{\underline{\vct{y}}}

%------------------------------------------------

\newcommand{\loF}{\linop{F}}
\newcommand{\loG}{\linop{G}}
\newcommand{\loH}{\linop{H}}

%------------------------------------------------

\newcommand{\setA}{\set{A}}
\newcommand{\setB}{\set{B}}
\newcommand{\setC}{\set{C}}
\newcommand{\setD}{\set{D}}
\newcommand{\setE}{\set{E}}
\newcommand{\setF}{\set{F}}
\newcommand{\setG}{\set{G}}
\newcommand{\setH}{\set{H}}
\newcommand{\setI}{\set{I}}
\newcommand{\setJ}{\set{J}}
\newcommand{\setK}{\set{K}}
\newcommand{\setL}{\set{L}}
\newcommand{\setM}{\set{M}}
\newcommand{\setN}{\set{N}}
\newcommand{\setO}{\set{O}}
\newcommand{\setP}{\set{P}}
\newcommand{\setQ}{\set{Q}}
\newcommand{\setR}{\set{R}}
\newcommand{\setS}{\set{S}}
\newcommand{\setT}{\set{T}}
\newcommand{\setU}{\set{U}}
\newcommand{\setV}{\set{V}}
\newcommand{\setW}{\set{W}}
\newcommand{\setX}{\set{X}}
\newcommand{\setY}{\set{Y}}
\newcommand{\setZ}{\set{Z}}


\begin{document}


\section{Convolutional Dictionary Model}

I will treat "convolution" and "circular convolution" as equivalent here. They are equivalent except at the boundaries. For boundary handling, see \cite{wohlberg2016boundary}.

We model a signal as a sum of convolutions. $\vk_{c,i}$ are the dictionary atom filters, $c$ specifying channel, and $i$ specifying the filter. $\vs_{c,n}$ is the signal, $c$ specifying channel, and $n$ specifying sample. $\vx_{i,n}$ are the coefficients, $i$ specifying the filter, and $n$ specifying the sample.

\begin{equation}
\vs_{c,n} = \vepsilon + \sum_{i} \vx_{i,n} * \vk_{c,i}
\end{equation}


We can construct a dictionary atom filter matrix by padding the filter, and then constructing a circulant matrix from the padded filter.

\begin{equation}
\mD_{c,i} = \operatorname{circ}(\operatorname{pad}(\vk_{c,i}))
\end{equation}

This matrix operator performs circular convolution:

\begin{equation}
\vx_{i,n} * \vk_{c,i} = \mD_{c,i}\vx_{i,n}
\end{equation}

So, we have the following.
\begin{equation}
\mD_c = \begin{bmatrix} \mD_{c,1} & \hdots & \mD_{c,K} \end{bmatrix}
\end{equation}

\begin{equation}
\vx_n = \begin{bmatrix} \vx_{1,n} \\
                      \vdots \\ 
                      \vx_{K,n} 
      \end{bmatrix}
\end{equation}

\begin{equation}
\vs_{c,n} = \vepsilon + \mD_c\vx_n
\end{equation}


$C$ is the number of channels.
$N$ is the number of samples.
$\vs_{i,n}$ is the $i$th channel of the $n$th sample signal.

\begin{equation}
\begin{aligned}
\mS = \begin{bmatrix} \vs_{1,1} & \hdots & \vs_{1,N} \\
                      \vdots & \ddots & \vdots \\
                      \vs_{C,1} & \hdots & \vs_{C,N}
      \end{bmatrix}
\end{aligned}
\end{equation}

$\mD_{c,i}$ is the $c$th channel of subdictionary corresponding to the $i$th dictionary atom filter.

\begin{equation}
\begin{aligned}
\mD = \begin{bmatrix} \mD_{1,1} & \hdots & \mD_{1,K} \\
                      \vdots & \ddots & \vdots \\
                      \mD_{C,1} & \hdots & \mD_{C,K}
      \end{bmatrix}
\end{aligned}
\end{equation}

$\vx_{i,n}$ is the coefficients corresponding to the $i$th dictionary atom filter for the $n$th sample.

\begin{equation}
\mX = \begin{bmatrix} \vx_{1,1} & \hdots & \vx_{1,N} \\
                      \vdots & \ddots & \vdots \\ 
                      \vx_{K,1} & \hdots & \vx_{K,N} 
      \end{bmatrix}
\end{equation}

Finally, we have the full equation:

\begin{equation}
\mS \approx \mD\mX
\end{equation}

\section{Dictionary Learning Overview}

Using the dictionary $\mD$ constructed from dictionary atom filters $\vk_i$, dictionary learning can be thought of as the following minimization problem:

\begin{equation}
\begin{aligned}
\min_{\mX,\mK} & \frac{1}{2}\|\mS - \mD\mX\|_F^2 + \lambda\|\mX\|_1\\
\text{subject to } & \|\vk_{c,i}\|_2 \leq 1\\
                   & \mD_{c,i} = \operatorname{circ}(\operatorname{pad}(\vk_{c,i}))
\end{aligned}
\end{equation}

Solving this jointly is just too hard. So, instead, we alternate updates of $\mD$ for fixed $\mX$ and updates for $\mX$ for fixed $\mD$.

\section{The Coefficient Update}

We can update coefficients independently for each sample:
\begin{equation}
\begin{aligned}
\min_{\vx_n} & \frac{1}{2}\|\vs_n - \mD\vx_n\|_F^2 + \lambda\|\vx_n\|_1
\end{aligned}
\end{equation}

The standard approach to solving this problem is alternating direction method of multipliers (ADMM).

\subsection{ADMM}

ADMM works like this. To solve the problem

\begin{equation}
\min_x f(x) + g(x)
\end{equation}

ADMM splits $\vx$ into two variables $\vx$ and $\vy$, and then constains the two to be equal.

\begin{equation}
\begin{aligned}
\min_{x,y} & f(x) + g(y) \\
\text{subject to } & y - x = 0
\end{aligned}
\end{equation}

ADMM is an iterative algorithm with the following update equations:

\begin{equation}
x^{(k + 1)} = \min_x f(x) + g(y^{(k)}) + \frac{\rho}{2}\|y^{(k)} - x + \frac{\gamma^{(k)}}{\rho}\|_2^2
\end{equation}

\begin{equation}
y^{(k + 1)} = \min_y f(x^{(k + 1)}) + g(y) + \frac{\rho}{2}\|y - x^({k + 1}) + \frac{\gamma^{(k)}}{\rho}\|_2^2
\end{equation}

\begin{equation}
\gamma^{(k + 1)} = \gamma^{(k)} + \rho(y^{(k + 1)} - x^{(k + 1)})
\end{equation}

Obviously, a couple variables were added here. $\rho$ is a scalar hyperparameter. $\gamma$ is the dual variable corresponding to the constraint $y - x = 0$. If you want a surface level understanding without delving deep into convex optimization, I think my ADMM slides are your best bet. Here's another source on ADMM \cite{boyd2011distributed}, but I don't think it's very accessable without a convex optimization background. 

\subsection{Applying ADMM to the Coefficient Update}

\begin{equation}
f(\vx_n) = \frac{1}{2}\|\vs_n - \mD\vx_n\|_F^2
\end{equation}


\begin{equation}
g(\vy_n) = \lambda\|\vy_n\|_1
\end{equation}

\begin{equation}
\begin{aligned}
\min_{\vx_n,\vy_n} & f(\vx_n) + g(\vy_n) \\
\text{subject to } & \vy_n - \vx_n = 0
\end{aligned}
\end{equation}

\subsubsection{First Update Equation}
\begin{equation}
\vx_n^{(k + 1)} = \min_{\vx} f(\vx) + g(\vy_n^{(k)}) + \frac{\rho}{2}\|\vy_n^{(k)} - \vx + \frac{\vgamma^{(k)}}{\rho}\|_2^2
\end{equation}

\begin{equation}
\vx_n^{(k + 1)} = \min_{\vx} \frac{1}{2}\|\vs_n - \mD\vx\|_F^2 + \lambda\|\vy_n^{(k)}\|_1 + \frac{\rho}{2}\|\vy_n^{(k)} - \vx + \frac{\vgamma^{(k)}}{\rho}\|_2^2
\end{equation}

Take the derivative in respect to $\vx$ and set equal to zero.

\begin{equation}
\vzero = \mD^T\mD\vx - \mD^T\vs_n + \rho\vx - \rho(\vy_n^{(k)}  + \frac{\vgamma^{(k)}}{\rho})
\end{equation}

Shuffling things around, we have

\begin{equation}
(\rho\mId + \mD^T\mD)\vx = \mD^T\vs_n + \rho(\vy_n^{(k)}  + \frac{\vgamma^{(k)}}{\rho})
\end{equation}

\begin{equation}
\vx_n^{(k + 1)} = (\rho\mId + \mD^T\mD)^{-1}(\mD^T\vs_n + \rho(\vy_n^{(k)}  + \frac{\vgamma^{(k)}}{\rho}))
\end{equation}

\subsubsection{Second Update equation}
\begin{equation}
\vy_n^{(k + 1)} = \min_{\vy} f(\vx_n^{(k + 1)}) + g(\vy) + \frac{\rho}{2}\|\vy - \vx_n^{(k + 1)} + \frac{\vgamma^{(k)}}{\rho}\|_2^2
\end{equation}

\begin{equation}
\vy_n^{(k + 1)} = \min_{\vy} \frac{1}{2}\|\vs_n - \mD\vx_n^{(k + 1)}\|_F^2 + \lambda\|\vy\|_1 + \frac{\rho}{2}\|\vy - \vx_n^{(k + 1)} + \frac{\vgamma^{(k)}}{\rho}\|_2^2
\end{equation}

Removing the irrelevant term.

\begin{equation}
\vy_n^{(k + 1)} = \min_{\vy}  \frac{\rho}{2}\|\vy - \vx_n^{(k + 1)} + \frac{\vgamma^{(k)}}{\rho}\|_2^2 + \lambda\|\vy\|_1
\end{equation}

This minimization problem is the proximal operator for the L1-norm. The solution is well-known, so I'll pass on deriving it. It's the elementwise shrinkage operator. (I included a brief explanation of proximal operators and FISTA at the end).

\begin{equation}
\operatorname{S}_{\lambda}(x) = \begin{cases}
x - \lambda & x > \lambda \\
0 & -\lambda < x < \lambda \\
x + \lambda & x < - \lambda
\end{cases}
\end{equation}

\begin{equation}
\vy_n^{(k + 1)} = S_{\lambda}(\vx_n^{(k + 1)} - \frac{\vgamma^{(k)}}{\rho})
\end{equation}

\subsubsection{Third Update Equation}

Nothing really changed here.

\begin{equation}
\vgamma^{(k + 1)} = \vgamma^{(k)} + \rho(\vy_n^{(k + 1)} - \vx_n^{(k + 1)})
\end{equation}

\subsection{Working with the Update Equations}

The second and third update equations are simple and straightforward to implement.

The first is not.

\begin{equation}
\vx_n^{(k + 1)} = (\rho\mId + \mD^T\mD)^{-1}(\mD^T\vs_n + \rho(\vy_n^{(k)}  + \frac{\vgamma^{(k)}}{\rho}))
\end{equation}

$\mD$ is a very large matrix. This inverse is impractical to compute directly.

Let's recall what $\mD$ is, though.

\begin{equation}
\begin{aligned}
\mD = \begin{bmatrix} \mD_{1,1} & \hdots & \mD_{1,K} \\
                      \vdots & \ddots & \vdots \\
                      \mD_{C,1} & \hdots & \mD_{C,K}
      \end{bmatrix}
\end{aligned}
\end{equation}


where

\begin{equation}
\mD_{c,i} = \operatorname{circ}(\operatorname{pad}(\vk_{c,i}))
\end{equation}

We're in luck here. The vectors of the Fourier basis are eigenvectors of circulant matricies, and circulant matrices are diagonalizable.

\begin{equation}
\begin{aligned}
\mD = \begin{bmatrix} \mathcal{F}^{-1}\hat{\mD}_{1,1}\mathcal{F} & \hdots & \mathcal{F}^{-1}\hat{\mD}_{1,K}\mathcal{F} \\
                      \vdots & \ddots & \vdots \\
                      \mathcal{F}^{-1}\hat{\mD}_{C,1}\mathcal{F} & \hdots & \mathcal{F}^{-1}\hat{\mD}_{C,K}\mathcal{F}
      \end{bmatrix}
\end{aligned}
\end{equation}

where $\hat{\mD}_{c,i}$ is diagonal.

Now, let's pull the fourier transforms outside the dictionary matrix.

\begin{equation}
\begin{aligned}
\hat{\mD} = \begin{bmatrix} \hat{\mD}_{1,1} & \hdots & \hat{\mD}_{1,K} \\
                      \vdots & \ddots & \vdots \\
                      \hat{\mD}_{C,1} & \hdots & \hat{\mD}_{C,K}
      \end{bmatrix}
\end{aligned}
\end{equation}

$\hat{\mD}$ is sparsely banded. Now, let's take a look at the matrix we are trying to invert.


\begin{equation}
\rho\mId + \mD^T\mD = \rho\mId + (\mathcal{F}^{-1}\hat{\mD}\mathcal{F})^T(\mathcal{F}^{-1}\hat{\mD}\mathcal{F})
\end{equation}
 
\begin{equation}
\rho\mId + \mD^T\mD = \rho\mId + \mathcal{F}^{-1}\hat{\mD}^H\mathcal{F}\mathcal{F}^{-1}\hat{\mD}\mathcal{F}
\end{equation}

\begin{equation}
\rho\mId + \mD^T\mD = \rho\mId + \mathcal{F}^{-1}\hat{\mD}^H\hat{\mD}\mathcal{F}
\end{equation}

Remember that $\mId = \mathcal{F}^{-1}\mathcal{F}$. So,

\begin{equation}
\rho\mId + \mD^T\mD = \mathcal{F}^{-1}(\rho\mId + \hat{\mD}^H\hat{\mD})\mathcal{F}
\end{equation}

\begin{equation}
(\rho\mId + \mD^T\mD)^{-1} = \mathcal{F}^{-1}(\rho\mId + \hat{\mD}^H\hat{\mD})^{-1}\mathcal{F}
\end{equation}

Looking closely at $\hat{\mD}$ and $\hat{\mD}^H\hat{\mD}$ we can see that frequency bands do not interact. That is, if a row or column contains a DC (direct current aka zero-frequency) component of any filter, it contains no other frequency components from any filter. Therefore, we can split the inverse problem into a single $K \times K$ inverse computation for each pixel or element in the signal.

Let $\hat{\mD}_{c,i}[j]$ be the $j$th diagonal element of $\hat{\mD}_{c,i}$. Then, let $\hat{\mD}[i]$ be the submatrix containing the $j$th diagonal elements for all channels and filters.

The inverse we want to compute is $(\rho\mId + \hat{\mD}^H[j]\hat{\mD}[j])^{-1}$.

Note that the rank of $\hat{\mD}^H[j]\hat{\mD}[j]$ is less than or equal to the number of channels. If this number is small, the inverse can be computed efficiently using the Woodbury matrix identity. Wikipedia has a great page on the Woodbury matrix identity if you are not familiar.

If you are looking for more explanations or sources of what I've explained so far, \cite{bristow2013fast} uses ADMM for convolutional sparse coding, \cite{heide2015fast} and \cite{vsorel2016fast} exploit the sum-of-low-rank-and-identity nature of the matrix to be inverted, and \cite{wohlberg2016boundary} and \cite{heide2015fast} dicuss how to handle boundaries.

But what if the number of channels is not small?

\subsection{Coefficient Update for Multi-Channel Signals}

If the number of channels is large, standard approaches offer two options (see \cite{garcia2018convolutional} (section VI)):
\begin{enumerate}
\item ADMM consensus:

In the past I have used the subscript of the coefficients and signals to refer to the sample.  Here, assume sample $n$, because I need the subscript to indicate channel. In the first update equation, estimate coefficients separately for each channel. For the second update equation, combine the results to get a single set of coefficients.

\begin{equation}
f_c(\vx_c) = \frac{1}{2}\|\vs_c - \mD\vx_c\|_F^2
\end{equation}

\begin{equation}
g(\vy) = \lambda\|\vy\|_1
\end{equation}


\begin{equation}
\begin{aligned}
\min_{\vx_c,\vy} & \sum_{c = 1}^C f_c(\vx_c) + g(\vy)\\
\text{subject to } & \vy - \vx_c = 0 & \forall c
\end{aligned}
\end{equation}

The update equations become the following:

Obviously, $C$ separate updates here.
\begin{equation}
\vx_c^{(k + 1)} = (\rho\mId + \mD_c^T\mD_c)^{-1}(\mD_c^T\vs_c + \rho(\vy^{(k)}  + \frac{\vgamma_c^{(k)}}{\rho}))
\end{equation}

\begin{equation}
\vy^{(k + 1)} = S_{\lambda}( \frac{1}{C}\sum_{c = 1}^C \vx_c^{(k + 1)} - \frac{\vgamma_c^{(k)}}{\rho})
\end{equation}

Obviously, $C$ separate updates here.
\begin{equation}
\vgamma_c^{(k + 1)} = \vgamma_c^{(k)} + \rho(\vy^{(k + 1)} - \vx_c^{(k + 1)})
\end{equation}

\item
Alternatively, you can use the FISTA algorithm \cite{beck2009fast}.
\end{enumerate}

Now, here is my thought on all this. I think we can do better than FISTA or ADMM consensus.  When learning a dictionary using an online method, the dictionary will be updated frequently. What if we could figure out a way to make the updates for each frequency low rank? Then we could use the Sherman-Morrison formula to efficiently update our inverses! Alternatively, a low-rank approximation could suffice, but I'm less enthused about that approach.

\section{The Dictionary Update}

Most dictionary update approaches are batch methods (they handle the entire dataset at once.)


Recall our minimization problem.

\begin{equation}
\begin{aligned}
\min_{\mX,\mK} & \frac{1}{2}\|\mS - \mD\mX\|_F^2 + \lambda\|\mX\|_1\\
\text{subject to } & \|\vk_i\|_2 \leq 1\\
                   & \mD_{c,i} = \operatorname{circ}(\operatorname{pad}(\vk_{c,i}))
\end{aligned}
\end{equation}

We are going to keep $\mX$ fixed, and update $\mD$.

\begin{equation}
\begin{aligned}
\min_{\mK} & \frac{1}{2}\|\mS - \mD\mX\|_F^2\\
\text{subject to } & \|\vk_i\|_2 \leq 1\\
                   & \mD_{c,i} = \operatorname{circ}(\operatorname{pad}(\vk_{c,i}))
\end{aligned}
\end{equation}

Before, we used the dictionary atom filters to generate circulant matrices for the dictionary. Now, we can do the same with the coefficients. To avoid confusion, we can call these coefficients $\mY$.

I am flipping $n$ and $i$ for this.

\begin{equation}
\mY_{n,i} = \operatorname{circ}(\vx_{i,n})
\end{equation}


\begin{equation}
\begin{aligned}
\mY = \begin{bmatrix} \mY_{1,1} & \hdots & \mY_{1, K}\\
                      \vdots & \ddots & \vdots \\
                      \mY_{N,1} & \hdots & \mY_{N,K}
      \end{bmatrix}
\end{aligned}
\end{equation}

\begin{equation}
\vd_c = \begin{bmatrix} \operatorname{pad}(\vk_{c,1}) \\
                        \vdots \\
                        \operatorname{pad}(\vk_{c,K})
         \end{bmatrix}
\end{equation}

\begin{equation}
\vs_c = \begin{bmatrix} \operatorname{pad}(\vs_{c,1}) \\
                        \vdots \\
                        \operatorname{pad}(\vs_{c,N})
         \end{bmatrix}
\end{equation}

\begin{equation}
\begin{aligned}
\min_{\vd_{c,i}} & \frac{1}{2} \sum_{c = 1}^C\|\vs_c - \mY\vd_c\|_2^2\\
\text{subject to } & \|\vk_{c,i}\|_2 \leq 1\\
                   & \vd_{c,i} = \operatorname{pad}(\vk_{c,i})
\end{aligned}
\end{equation}

The two constraints
\begin{equation}
\|\vk_{c,i}\|_2 \leq 1
\end{equation}

and 

\begin{equation}
\vd_{c,i} = \operatorname{pad}(\vk_{c,i})
\end{equation}

are convex constraints, the first normalizing the filter, and the second requiring the filters to be small/local. Let $T$ be the convex set the constraints require $\vd_{c,i}$ to be in.

The simplest approach to an online algorithm is projected gradient descent.

That would look like this:

\begin{equation}
\vd_c^{(k + 1)} = \operatorname{proj}_{T}(\vd_c^{(k)} - \alpha\mY_n^T(\mY_n\vd_{c}^{(k)} - \vs_n))
\end{equation}

where $\alpha$ is the step size.

There are a couple variants of this, but ultimately things end up very similar. Add momentum, and you get the FISTA algorithm \cite{beck2009fast}. Include past coefficients in your gradient calculation before projection, and you get a second-order method \cite{liu2018first}.

My question is, can we find an update close to this one that is low-rank for each frequency, for efficient inverse updates? It feels like there has to be a better approach to coefficient updates than FISTA or consensus ADMM when doing online dictionary learning for multi-channel data.

\section{FISTA}

I've mentioned the fast iterative shrinkage algorithm a few times, so I probably should go ahead and explain it.

Suppose you have a two-term objective function $f(x) + g(x)$ and you want to minimize it.

$f$ is easily differentiable, and g has a simple proximal operator.

Definition of proximal operator:
\begin{equation}
\operatorname{prox}_g(x) = \arg \min_y \frac{1}{2}\|y - x\|_2^2 + g(y)
\end{equation}

Then, the iterative shrinkage thresholding algorithm (ISTA) goes like this:

\begin{equation}
\vx^{(k + 1)} = \operatorname{prox}_g(\vx^{(k)} - \alpha\nabla_{\vx}f(\vx^{(k)})
\end{equation}

Where $\alpha$ is step size for gradient steps.

The fast iterative shrinkage thresholding algorithm (FISTA) just adds momentum to the mix.

\begin{equation}
\vy^{(k + 1)} = \operatorname{prox}_g(\vx^{(k)} - \alpha\nabla_{\vx}f(\vx^{(k)})
\end{equation}

\begin{equation}
t^{(k + 1)} = \frac{1}{2} (1 + \sqrt{1 + 4(t^{(k)})^2})
\end{equation}

\begin{equation}
\vx^{(k + 1)} = \vy^{(k + 1)} + \frac{t^{(k)} - 1}{t^{(k + 1)}}(z^{(k + 1)} - \vx^{(k)})
\end{equation}

The concept of momentum (in case you aren't familiar) is a gradient descent concept. Instead of computing the gradient at the current location, you compute the gradient a little further ahead.

In the context of the equations above, think of $\vz$ as your current location, but $\vx$ as where you are looking for gradient information.

\bibliography{onlineCDL-lrupdatesSources}
\bibliographystyle{ieeetr}
\end{document}
