\documentclass{article}
\usepackage{amsmath}
\usepackage{bm}
\usepackage{bbm}

% fields
\newcommand{\R}{\mathbb{R}}
\newcommand{\C}{\mathbb{C}}
\newcommand{\Z}{\mathbb{Z}}
\newcommand{\N}{\mathbb{N}}

% complex numbers
\renewcommand{\Re}[1]{\operatorname{Re}\left\{#1\right\}}
\renewcommand{\Im}[1]{\operatorname{Im}\left\{#1\right\}}
\newcommand{\conj}[1]{\mkern 1.5mu\overline{\mkern-1.5mu#1\mkern-1.5mu}\mkern 1.5mu}

% probability and stat
\renewcommand{\P}[1]{\operatorname{P}\left(#1\right)}
\newcommand{\E}{\operatorname{E}}
\newcommand{\var}{\operatorname{var}}

% calculus 
\renewcommand{\d}[1]{d#1}

% constants (written in roman, if wanted)
\newcommand{\e}{e}
\renewcommand{\j}{j}

% linear algebra
% 	vector notation
%\newcommand{\vct}[1]{\boldsymbol{#1}}
\newcommand{\vct}[1]{\bm{#1}}
%   matrices
%\newcommand{\mtx}[1]{\boldsymbol{#1}}
\newcommand{\mtx}[1]{\bm{#1}}
% diagonal
\newcommand{\diag}{\operatorname{diag}}
%   block vector
\newcommand{\bvct}[1]{\mathbf{#1}}
%   block matrix
\newcommand{\bmtx}[1]{\mathbf{#1}}
%	inner products
\newcommand{\<}{\langle}
\renewcommand{\>}{\rangle}
% 	transpose, Hermitian, pseudo-inverse
\renewcommand{\H}{\mathrm{H}}
\newcommand{\T}{\mathrm{T}}
\newcommand{\pinv}{\dagger}
%	fundamental subspaces
\newcommand{\Null}{\operatorname{Null}}
\newcommand{\Range}{\operatorname{Range}}
\newcommand{\Span}{\operatorname{Span}}
%	operators
\newcommand{\trace}{\operatorname{trace}}
\newcommand{\rank}{\operatorname{rank}}
\newcommand{\F}{\mathcal{F}}

% sets and topology
\newcommand{\set}[1]{\mathcal{#1}}
\newcommand{\closure}{\operatorname{cl}}	% closure
\newcommand{\interior}{\operatorname{int}}
\newcommand{\boundary}{\operatorname{bd}}
\newcommand{\diameter}{\operatorname{diam}}

% functional analysis
\newcommand{\domain}{\operatorname{dom}}
\newcommand{\epigraph}{\operatorname{epi}}
\newcommand{\hypograph}{\operatorname{hypo}}
\newcommand{\linop}[1]{\mathscr{#1}}	% general linear operator

% optimization
\renewcommand{\L}{\mathcal{L}}
\DeclareMathOperator*{\minimize}{\text{minimize}}
\DeclareMathOperator*{\maximize}{\text{maximize}}
\newcommand{\indicator}{\mathbbm{1}}
\newcommand{\prox}{\operatorname{prox}}

% Complexity
\newcommand{\bigO}{\mathcal{O}}

% Neural Networks
\newcommand{\pool}{\mtx{P}}


%--------------------------------------------------------------------------


\newcommand{\va}{\vct{a}}
\newcommand{\vb}{\vct{b}}
\newcommand{\vc}{\vct{c}}
\newcommand{\vd}{\vct{d}}
\newcommand{\ve}{\vct{e}}
\newcommand{\vf}{\vct{f}}
\newcommand{\vg}{\vct{g}}
\newcommand{\vh}{\vct{h}}
\newcommand{\vi}{\vct{i}}
\newcommand{\vj}{\vct{j}}
\newcommand{\vk}{\vct{k}}
\newcommand{\vl}{\vct{l}}
\newcommand{\vm}{\vct{m}}
\newcommand{\vn}{\vct{n}}
\newcommand{\vo}{\vct{o}}
\newcommand{\vp}{\vct{p}}
\newcommand{\vq}{\vct{q}}
\newcommand{\vr}{\vct{r}}
\newcommand{\vs}{\vct{s}}
\newcommand{\vt}{\vct{t}}
\newcommand{\vu}{\vct{u}}
\newcommand{\vv}{\vct{v}}
\newcommand{\bvv}{\bvct{v}}
\newcommand{\vw}{\vct{w}}
\newcommand{\vx}{\vct{x}}
\newcommand{\vy}{\vct{y}}
\newcommand{\vz}{\vct{z}}
\newcommand{\bvz}{\bvct{z}}
%
\newcommand{\valpha}{\vct{\alpha}}
\newcommand{\bvalpha}{\bvct{\alpha}}
\newcommand{\vbeta}{\vct{\beta}}
\newcommand{\vepsilon}{\vct{\epsilon}}
\newcommand{\vgamma}{\vct{\gamma}}
\newcommand{\vlambda}{\vct{\lambda}}
\newcommand{\vnu}{\vct{\nu}}
\newcommand{\vmu}{\vct{\mu}}
\newcommand{\bvmu}{\bvct{\mu}}
\newcommand{\vphi}{\vct{\phi}}
\newcommand{\vpsi}{\vct{\psi}}
\newcommand{\vtheta}{\vct{\theta}}
\newcommand{\veta}{\vct{\eta}}
%
\newcommand{\vzero}{\vct{0}}
\newcommand{\vone}{\vct{1}}

%------------------------------------------------

\newcommand{\mA}{\mtx{A}}
\newcommand{\mB}{\mtx{B}}
\newcommand{\mC}{\mtx{C}}
\newcommand{\mD}{\mtx{D}}
\newcommand{\bmD}{\bmtx{D}}
\newcommand{\mE}{\mtx{E}}
\newcommand{\mF}{\mtx{F}}
\newcommand{\mG}{\mtx{G}}
\newcommand{\mH}{\mtx{H}}
\newcommand{\mJ}{\mtx{J}}
\newcommand{\mK}{\mtx{K}}
\newcommand{\mL}{\mtx{L}}
\newcommand{\mM}{\mtx{M}}
\newcommand{\mN}{\mtx{N}}
\newcommand{\mO}{\mtx{O}}
\newcommand{\mP}{\mtx{P}}
\newcommand{\mQ}{\mtx{Q}}
\newcommand{\mR}{\mtx{R}}
\newcommand{\mS}{\mtx{S}}
\newcommand{\mT}{\mtx{T}}
\newcommand{\mU}{\mtx{U}}
\newcommand{\mV}{\mtx{V}}
\newcommand{\mW}{\mtx{W}}
\newcommand{\mX}{\mtx{X}}
\newcommand{\mY}{\mtx{Y}}
\newcommand{\mZ}{\mtx{Z}}
%
\newcommand{\mDelta}{\mtx{\Delta}}
\newcommand{\mLambda}{\mtx{\Lambda}}
\newcommand{\mPhi}{\mtx{\Phi}}
\newcommand{\mPsi}{\mtx{\Psi}}
\newcommand{\mSigma}{\mtx{\Sigma}}
\newcommand{\mUpsilon}{\mtx{\Upsilon}}
\newcommand{\mGamma}{\mtx{\Gamma}}
%
\newcommand{\mId}{{\bf I}}
\newcommand{\mEx}{{\bf J}}
\newcommand{\mzero}{{\bf 0}}
\newcommand{\mone}{{\bf 1}}

\newcommand{\mAbar}{\underline{\mtx{A}}}
\newcommand{\mRbar}{\underline{\mtx{R}}}
\newcommand{\vebar}{\underline{\vct{e}}}
\newcommand{\vxbar}{\underline{\vct{x}}}
\newcommand{\vybar}{\underline{\vct{y}}}

%------------------------------------------------

\newcommand{\loF}{\linop{F}}
\newcommand{\loG}{\linop{G}}
\newcommand{\loH}{\linop{H}}

%------------------------------------------------

\newcommand{\setA}{\set{A}}
\newcommand{\setB}{\set{B}}
\newcommand{\setC}{\set{C}}
\newcommand{\setD}{\set{D}}
\newcommand{\setE}{\set{E}}
\newcommand{\setF}{\set{F}}
\newcommand{\setG}{\set{G}}
\newcommand{\setH}{\set{H}}
\newcommand{\setI}{\set{I}}
\newcommand{\setJ}{\set{J}}
\newcommand{\setK}{\set{K}}
\newcommand{\setL}{\set{L}}
\newcommand{\setM}{\set{M}}
\newcommand{\setN}{\set{N}}
\newcommand{\setO}{\set{O}}
\newcommand{\setP}{\set{P}}
\newcommand{\setQ}{\set{Q}}
\newcommand{\setR}{\set{R}}
\newcommand{\setS}{\set{S}}
\newcommand{\setT}{\set{T}}
\newcommand{\setU}{\set{U}}
\newcommand{\setV}{\set{V}}
\newcommand{\setW}{\set{W}}
\newcommand{\setX}{\set{X}}
\newcommand{\setY}{\set{Y}}
\newcommand{\setZ}{\set{Z}}


\begin{document}
This is the problem I'm trying to solve:

\begin{equation}
f(\vz) =  \frac{a}{2}\|\vz - \vx\|_2^2 + \frac{b}{2}\|\mP\vz - \mP\vs\|_2^2
\end{equation}

\begin{equation}
\arg\min_{\vz} f(\vz)
\end{equation}

where $\mP$ is a nonlinear operator, specifically a quantized projection operator.

That is, $\mP(\cdot) = \mW^T\operatorname{Quantize}(\mW\cdot)$, where $\mW^T\mW$ is a projection operator.

\begin{equation}
\operatorname{Quantize}(\vy) = \operatorname{round}(\frac{\vy}{\vq})*\vq
\end{equation}
(Division here is element-by-element).

In an early attempt to solve my problem, I pretended $\mP$ was a linear projection operator and came up with an approximate solution to my original problem.

\begin{equation}
\vz_{\text{approx}} = \vx + \frac{b}{a + b}(\mP\vs - \mP\vx)
\end{equation}

Now, plugging $\vz = \vz_{\text{approx}} + \Delta\vz$ back into the original equation, I have the expression:
\begin{equation}
f(\vz_{\text{approx}} + \Delta\vz) = \frac{a}{2}\|\frac{b}{a + b}(\mP\vs - \mP\vx) + \Delta\vz\|_2^2 + \frac{b}{2}\|\mP(\vx + \frac{b}{a + b}(\mP\vs - \mP\vx) + \Delta\vz) - \mP\vs\|_2^2
\end{equation}


In analyzing this expression, I have found it helpful to define a function $\vepsilon(\Delta\vz)$:
\begin{equation}
\vepsilon(\Delta\vz) = \mP(\vx + \frac{b}{a + b}(\mP\vs - \mP\vx) + \Delta\vz) - \mP\vx - \frac{b}{a + b}(\mP\vs - \mP\vx)
\end{equation}

\begin{equation}
\mP(\vx + \frac{b}{a + b}(\mP\vs - \mP\vx) + \Delta\vz) = \mP\vx - \frac{b}{a + b}(\mP\vs - \mP\vx) +\vepsilon(\Delta\vz)
\end{equation}

Returning to the objective function:
\begin{equation}
f(\vz_{\text{approx}} + \Delta\vz) = \frac{a}{2}\|\frac{b}{a + b}(\mP\vs - \mP\vx) + \Delta\vz\|_2^2 + \frac{b}{2}\|\mP\vx - \frac{b}{a + b}(\mP\vs - \mP\vx) +\vepsilon(\Delta\vz) - \mP\vs\|_2^2
\end{equation}

Simplifying
\begin{equation}
f(\vz_{\text{approx}} + \Delta\vz) = \frac{a}{2}\|\frac{b}{a + b}(\mP\vs - \mP\vx) + \Delta\vz\|_2^2 + \frac{b}{2}\|-\frac{a + b}{a + b}(\mP\vs - \mP\vx) + \frac{b}{a + b}(\mP\vs - \mP\vx) +\vepsilon(\Delta\vz)\|_2^2
\end{equation}

\begin{equation}
f(\vz_{\text{approx}} + \Delta\vz) = \frac{a}{2}\|\frac{b}{a + b}(\mP\vs - \mP\vx) + \Delta\vz\|_2^2 + \frac{b}{2}\|- \frac{a}{a + b}(\mP\vs - \mP\vx) +\vepsilon(\Delta\vz)\|_2^2
\end{equation}

I have $2$ objective terms, and it will help to give them names:
\begin{equation}
f_1(\Delta\vz) = \frac{a}{2}\|\frac{b}{a + b}(\mP\vs - \mP\vx) + \Delta\vz\|_2^2
\end{equation}

\begin{equation}
f_2(\Delta\vz) = \frac{b}{2}\|- \frac{a}{a + b}(\mP\vs - \mP\vx) +\vepsilon(\Delta\vz)\|_2^2
\end{equation}

\begin{equation}
f(\vz) = f_1(\vz - \vz_{\text{approx}}) + f_2(\vz - \vz_{\text{approx}})
\end{equation}

Now for some observations:
\begin{enumerate}
\item
Adding a component to $\Delta\vz$ that is orthogonal to the span of the columns of $\mW^T$ increases the first term of the objective $f_1$ without affecting the second term $f_2$.
\begin{equation}
f_1(\Delta\vz) \geq f_1(\mW^T\mW\Delta\vz)
\end{equation}

\begin{equation}
f_2(\Delta\vz) = f_2(\mW^T\mW\Delta\vz)
\end{equation}

Therefore,
\begin{equation}
(\mId - \mW^T\mW)(\Delta\vz)_{\text{optimal}} = 0
\end{equation}

\item
In the simplified case of no quantization $\mP = \mW^T\mW$:
\begin{equation}
\vepsilon(\Delta\vz) = \mW^T\mW\Delta\vz
\end{equation}

And so, if the quantization process is removed,
\begin{equation}
(\Delta\vz)_{\text{optimal}} = 0
\end{equation}

\item
For $\alpha \in [0,1]$:
\begin{equation}
f_2(\alpha\vepsilon(0)) = f_2(0)
\end{equation}

Furthermore, this is true even if I scale elements of $\mW\vepsilon(0)$ individually:
For $\alpha_i \in [0,1]$:
\begin{equation}
f_2(\mW^T\operatorname{diag}(\valpha)\mW\vepsilon(0)) = f_2(0)
\end{equation}

\item
There are certain choices for $\valpha$ from the previous observation that will decrease the first objective term $f_1$:

\begin{equation}
\alpha_i = \begin{cases} 1 & \operatorname{sign}(\ve_i^T\mW\vepsilon(0)) = -\operatorname{sign}(\ve_i^T\mW(\mP\vs - \mP\vx)) \\
                         0 & \text{otherwise}
           \end{cases}
\end{equation}
Using the $\valpha$ defined above:
\begin{equation}
f_1(\mW^T\operatorname{diag}(\valpha)\mW\vepsilon(0)) \leq f_1(0)
\end{equation}

\item
The last couple of operations have focused on decreasing $f_1$ without affecting $f_2$. Here, I observe it is also possible to select a $\Delta\vz$ that deceases $f_2$ by more than it decreases $f_1$.

\begin{equation}
\beta_i = \begin{cases} 1 + \nu  & \operatorname{sign}(\ve_i^T\mW\vepsilon(\vq/2)) = \operatorname{sign}(\ve_i^T\mW(\mP\vs - \mP\vx)) \neq 0 \\
                         0 & \text{otherwise}
           \end{cases}
\end{equation}
where $\nu$ is an arbitrarily small number to ensure the rounding occurs in the proper direction.

Using the $\vbeta$ defined above:
\begin{equation}
f_1(\mW^T\operatorname{diag}(\vbeta)\mW\vepsilon(\frac{\mW^T\vq}{2})) - f_1(0) \leq f_2(0) - f_2(\mW^T\operatorname{diag}(\vbeta)\mW\vepsilon(\frac{\mW^T\vq}{2}))
\end{equation}

\item
Finally, the last couple observations can be combined for the optimal solution:
\begin{equation}
(\Delta\vz)_{\text{optimal}} = \mW^T\operatorname{diag}(\vbeta)\mW\vepsilon(\frac{\mW^T\vq}{2}) + \mW^T\operatorname{diag}(\valpha)\mW\vepsilon(0)
\end{equation}

where

\begin{equation}
\alpha_i = \begin{cases} 1 & \operatorname{sign}(\ve_i^T\mW\vepsilon(0)) = -\operatorname{sign}(\ve_i^T\mW(\mP\vs - \mP\vx)) \\
                         0 & \text{otherwise}
           \end{cases}
\end{equation}

\begin{equation}
\beta_i = \begin{cases} 1 + \nu  & \operatorname{sign}(\ve_i^T\mW\vepsilon(\vq/2)) = \operatorname{sign}(\ve_i^T\mW(\mP\vs - \mP\vx)) \neq 0 \\
                        0 & \text{otherwise}
          \end{cases}
\end{equation}
\begin{equation}
\vz_{\text{optimal}} = \vx + \frac{b}{a + b}(\mP\vs - \mP\vx) + \mW^T\operatorname{diag}(\vbeta)\mW\vepsilon(\frac{\mW^T\vq}{2}) + \mW^T\operatorname{diag}(\valpha)\mW\vepsilon(0)
\end{equation}

\end{enumerate}


I still need to solve a slight variation of the above problem. Hopefully, the solution can be found in a similar way.

\begin{equation}
f(\vz) =  \frac{a}{2}\|\vz - \vx\|_2^2 + \frac{b}{2}\|\mP\vz + (1 - \mu)\mP\vy - (2 - \mu)\mP\vs\|_2^2
\end{equation}

\begin{equation}
\arg\min_{\vz} f(\vz)
\end{equation}

\begin{equation}
\vz_{\text{approx}} = \vx + \frac{b}{a + b}((2 - \mu)\mP\vs - (1 - \mu)\mP\vy - \mP\vx)
\end{equation}

To prevent derivations from falling off the page:
\begin{equation}
\vr = (2 - \mu)\mP\vs - (1 - \mu)\mP\vy - \mP\vx
\end{equation}


\begin{equation}
f(\vz_{\text{approx}} + \Delta\vz) = \frac{a}{2}\|\frac{b}{a + b}\vr + \Delta\vz\|_2^2 + \frac{b}{2}\|\mP(\vx + \frac{b}{a + b}\vr + \Delta\vz) + (1 - \mu)\mP\vy - (2 - \mu)\mP\vs\|_2^2
\end{equation}

\begin{equation}
\vepsilon(\Delta\vz) = \mP(\vx + \frac{b}{a + b}((2 - \mu)\mP\vs - (1 - \mu)\mP\vy - \mP\vx) + \Delta\vz) - \mP\vx - \frac{b}{a + b}((2 - \mu)\mP\vs - (1 - \mu)\mP\vy - \mP\vx)
\end{equation}

\begin{equation}
\vepsilon(\Delta\vz) = \mP(\vx + \frac{b}{a + b}\vr + \Delta\vz) - \mP\vx - \frac{b}{a + b}\vr
\end{equation}

\begin{equation}
\mP(\vx + \frac{b}{a + b}\vr + \Delta\vz) = \mP\vx + \frac{b}{a + b}\vr  + \vepsilon(\Delta\vz)
\end{equation}

\begin{equation}
f(\vz_{\text{approx}} + \Delta\vz) = \frac{a}{2}\|\frac{b}{a + b}\vr + \Delta\vz\|_2^2 + \frac{b}{2}\|\mP\vx + \frac{b}{a + b}\vr  + \vepsilon(\Delta\vz) + (1 - \mu)\mP\vy - (2 - \mu)\mP\vs\|_2^2
\end{equation}

\begin{equation}
f(\vz_{\text{approx}} + \Delta\vz) = \frac{a}{2}\|\frac{b}{a + b}\vr + \Delta\vz\|_2^2 + \frac{b}{2}\|-\vr + \frac{b}{a + b}\vr  + \vepsilon(\Delta\vz)\|_2^2
\end{equation}

\begin{equation}
f(\vz_{\text{approx}} + \Delta\vz) = \frac{a}{2}\|\frac{b}{a + b}\vr + \Delta\vz\|_2^2 + \frac{b}{2}\| -\frac{a}{a + b}\vr  + \vepsilon(\Delta\vz)\|_2^2
\end{equation}

The important distinction here is that $\mu$ prevents $\vr$ from being quantized by $\mP$'s quantization.  So, rounding in the "right direction" could go too far. 

\end{document}
