\documentclass{article}
\usepackage{amsmath}
\usepackage{bm}
\usepackage{bbm}

\input{defs-OCDL-LRU}
\begin{document}

My approach is to use low-rank updates to a factorization of

\begin{equation}
\rho\mId + \mD^H\mD
\end{equation}

for inverse computations. Alternatively, the form

\begin{equation}
\rho\mId + \mD\mD^H
\end{equation}

can be used exploiting Woodbury Matrix Lemma.

The low-rank update takes the form:

\begin{equation}
\mD_{k+1} = \mD_k + \vu\vv^H
\end{equation}

The low-rank update produces the following equation.

\begin{equation}
\rho\mId + (\mD + \vu\vv^H)^H(\mD + \vu\vv^H) = \rho\mId + \mD^H\mD + \vu^H\vu \vv\vv^H + \vv\vu^H\mD + \mD^H\vu\vv^H
\end{equation}

The expression
\begin{equation}
vvvu^H\mD + \mD^H\vu\vv
\end{equation}

produces a Hermitian matrix, so eigenvalues are real and eigenvectors can be chosen to be orthogonal, but there is no guarentee of positive definiteness. Negative eigenvalues aren't even rare.

However, given the original form:
\begin{equation}
\rho\mId + (\mD + \vu\vv^H)^H(\mD + \vu\vv^H) 
\end{equation}

it is obvious the system is positive definite. Therfore, we should be careful to compute the downdates last, and make sure the update algorithm can handle downdates without undermining stability.
\end{document}
