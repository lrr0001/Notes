\documentclass{article}
\usepackage{amsmath}
\usepackage{bm}
\usepackage{bbm}

\input{defs-OCDL-LRU}
\begin{document}

This is clever and all, but $\mB\mA$ has the same eigenvalues as $\mA\mB$, and if $\vy$ is an eigenvector of $\mB\mA$, then $\mA\vy$ is an eigenvector of $\mA\mB$. Much simpler methods exist to solve this problem. Still interested in the complicated way? Read on.


\begin{equation}
\vu\vv^H + \vv\vu^H = \lambda_1 \vx\vx^H + \lambda_2 \vy\vy^H
\end{equation}

Task:
Given $\vu$ and $\vv$, find $\lambda_1$, $\lambda_2$, $\vx$, and $\vy$

Let $\mA = \vu\vv^H + \vv\vu^H$

\begin{equation}
\mA\vu = (\vv^H\vu)\vu + (\vu^H\vu)\vv
\end{equation}

\begin{equation}
\mA\vv = (\vv^H\vv)\vu + (\vu^H\vv)\vv
\end{equation}

\begin{equation}
\mA^2\vu = (\vv^H\vu)((\vv^H\vu)\vu + (\vu^H\vu)\vv) + (\vu^H\vu)((\vv^H\vv)\vu + (\vu^H\vv)\vv)
\end{equation}

\begin{equation}
\mA^2\vu = ((\vv^H\vu)(\vv^H\vu) + (\vu^H\vu)(\vv^H\vv))\vu + ((\vv^H\vu)(\vu^H\vu) + (\vu^H\vu)(\vu^H\vv))\vv
\end{equation}

\begin{equation}
\mA^2\vu = ((\vv^H\vu)^2 + (\vu^H\vu)(\vv^H\vv))\vu + (2\Re{\vv^H\vu}(\vu^H\vu))\vv
\end{equation}

\begin{equation}
\mA^2\vu = \begin{bmatrix} \vu & \mA\vu \end{bmatrix} \begin{bmatrix} -c \\ -b \end{bmatrix}
\end{equation}

Need to solve for $b$ and $c$.

\begin{equation}
\begin{bmatrix}
\vu & \mA\vu
\end{bmatrix} =
\begin{bmatrix}
\vu & \vv
\end{bmatrix}
\begin{bmatrix}
1 & \vv^H\vu\\
0 & \vu^H\vu
\end{bmatrix}
\end{equation}

\begin{equation}
\begin{bmatrix}
\vu & \vv
\end{bmatrix} =
\begin{bmatrix}
\vu & \mA\vu
\end{bmatrix}
\frac{1}{\vu^H\vu}
\begin{bmatrix}
\vu^H\vu & -\vv^H\vu\\
0 & 1
\end{bmatrix}
\end{equation}

\begin{equation}
\mA^2\vu = \begin{bmatrix} \vu & \vv \end{bmatrix} \begin{bmatrix} (\vv^H\vu)^2 + (\vu^H\vu)(\vv^H\vv) \\ 2\Re{\vv^H\vu}(\vu^H\vu) \end{bmatrix}
\end{equation}

\begin{equation}
\mA^2\vu = \begin{bmatrix}
\vu & \mA\vu
\end{bmatrix}
\frac{1}{\vu^H\vu}
\begin{bmatrix}
\vu^H\vu & -\vv^H\vu\\
0 & 1
\end{bmatrix}
\begin{bmatrix} (\vv^H\vu)^2 + (\vu^H\vu)(\vv^H\vv) \\ 2\Re{\vv^H\vu}(\vu^H\vu) \end{bmatrix}
\end{equation}

\begin{equation}
\mA^2\vu = \begin{bmatrix}
\vu & \mA\vu
\end{bmatrix}
\begin{bmatrix} (\vv^H\vu)^2 + (\vu^H\vu)(\vv^H\vv) - 2\Re{\vv^H\vu}\vv^H\vu \\ 2\Re{\vv^H\vu} \end{bmatrix}
\end{equation}

\begin{equation}
\lambda^2 + b\lambda + c = 0
\end{equation}

\begin{equation}
b = -2\Re{\vv^H\vu}
\end{equation}

\begin{equation}
c = 2\Re{\vv^H\vu}\vv^H\vu - (\vu^H\vu)(\vv^H\vv) - (\vv^H\vu)^2
\end{equation}

\begin{equation}
\lambda = \Re{\vv^H\vu} \pm \frac{1}{2}\sqrt{\Re{\vv^H\vu}^2 + 4( (\vv^H\vu)^2 + (\vu^H\vu)(\vv^H\vv) - 2\Re{\vv^H\vu}\vv^H\vu)}
\end{equation}

This is clever and all, but $\mB\mA$ has the same eigenvalues as $\mA\mB$, and if $\vy$ is an eigenvector of $\mB\mA$, then $\mA\vy$ is an eigenvector of $\mA\mB$.

\end{document}
